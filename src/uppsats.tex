\documentclass{kththesis}
\usepackage{csquotes} % Recommended by biblatex
\usepackage[style=numeric,sorting=none,backend=biber]{biblatex}
\usepackage[swedish]{babel}
\addbibresource{references.bib} % The file containing our references, in BibTeX format

\title{English title}
\alttitle{Detta är den svenska översättningen av titeln}
\author{Oscar Råhlén och Sacharias Sjöqvist}
\email{orahlen@kth.se och sacsjo@kth.se}
\supervisor{Handledare}
\examiner{Examinator}
\programme{Degree project in Computer Science}
\school{School of Electrical Engineering and Computer Science}
\date{\today}

% Uncomment the next line to include cover generated at https://intra.kth.se/kth-cover?l=en
% \kthcover{kth-cover.pdf}

\begin{document}
% Frontmatter includes the title-page, abstracts and table-of-contents
\frontmatter

\titlepage

\begin{abstract}
  English abstract goes here.
\end{abstract}

\begin{otherlanguage}{swedish}
  \begin{abstract}
    Svenskt sammanfattning
  \end{abstract}
\end{otherlanguage}

\tableofcontents

% Mainmatter is where the actual contents of the thesis goes
\mainmatter

% Citera med \texttt{} och \parencite{}

\chapter{Introduktion}
Information om ämnet, leda in läsaren samt förklara uppsatsens relevans \parencite{heisenberg2015}. 
Vi borde sikta på ungefär 40 sidor totalt.

  \section{Problemformulering}
  Vad är syftet med uppsatsen? Vad vill ni uppnå

  \section{Frågeställning}
  Vår frågeställning

  \section{Avgränsningar}
  Vilka avgränsningar vi gjort i datamängder, testpersoner samt modeller.

\chapter{Bakgrund}
Teorin som tillhör vårt område. Främst ergonomi, maskininlärning, djupinlärning och CNN. Tidigare forskning inom området. ANN. 
Alla olika tekniker vi kommer att använda. Relaterade arbeten. 

  \section{Maskininlärning}
  Supervised learning. Image classification.

\chapter{Metod}
Hur vi gått tillväga. Vilka dataset, hur implementation gått till (verktyg, klassificerare, parametrar), hur vi valt features.
Hur evalueringen har gått till (traning, test, validation set).

\chapter{Resultat}
Presentation av resultatet från våra olika tekniker och evalueringar. i tabeller och grafer. Även beräkningstid. 

\chapter{Diskussion}
Diskutera resultatet och hur olika delar kan ha påverkat eller påverkade. Diskutera eventuell framtida forskning. Begränsningar med resultatet.
Etiska aspekter. Hållbarhet. 

\chapter{Slutsats}
Slutsats av vad vi kom fram till.

\printbibliography[heading=bibintoc]
\appendix
  \chapter{Appendix A}

\tailmatter
\end{document}
